% ------------------------
% Declaração do documento |
% ------------------------
\documentclass{article}
% ----------------------------
% Pacotes usados no documento |
% ----------------------------
\usepackage[utf8]{inputenc}
\usepackage[left=2cm,top=2cm,right=3cm,bottom=3cm]{geometry}
\usepackage[ampersand]{easylist}
\usepackage{graphicx}
\usepackage{enumitem}
\usepackage{indentfirst}
\usepackage{float}
\usepackage[portuguese]{babel}
\usepackage{ifthen}
\usepackage{xcolor}
% ---------------------
% Customizações gerais |
% ---------------------
\newcounter{rncounter}
\newcounter{cscounter}
\renewcommand{\arraystretch}{2}
% --------------------
% Ambientes           |
% --------------------
\newcommand{\sumario}[1] {\textbf{Sumário:} #1\\ }
\newcommand{\ator}[1] {\textbf{Ator Primário:} #1\\}
\newcommand{\precond}[1] {\textbf{Precondições:} #1\\}
\newcommand{\regras}{\textbf{Regras de negócio: }}
\newcommand{\fluxo}{\textbf{Fluxo Principal:}}
\newcommand{\diagrama}[2]
{
 \begin{figure}[H]
 \begin{center}
 \includegraphics[width=\textwidth]{#1.png}
 \end{center}
 \caption{#2}
 \label{fig:#1}
 \end{figure}
}
\newcommand{\telasmart}[2]
{
 \begin{figure}[H]
 \begin{center}
 \includegraphics[scale=0.3]{#1.png}
 \end{center}
 \caption{#2}
 \label{fig:#1}
 \end{figure}
}

\newenvironment{fluxoa}[2]
	{
		\textbf{Fluxo Alternativo (#1): #2}
		\begin{enumerate}[itemsep=0mm, label=(\alph*)]			
	}
	{
		\end{enumerate}			
	}
\newenvironment{fluxoe}[2]
	{
		\textbf{Fluxo de Exceção (#1): #2}
		\begin{enumerate}[itemsep=0mm, label=(\alph*)]			
	}
	{
		\end{enumerate}			
	}

\newenvironment{casosdeuso}[1]
{
 \stepcounter{cscounter}
 \begin{center}
 \begin{tabular}{|p{\textwidth}|}
 \hline
 \begin{center}
 \large \textbf{#1 (CSU\ifnum\value{cscounter}<10 0\fi\arabic{cscounter})}
 \end{center}
}
{ 
 \\\\\hline
 \end{tabular} 
 \end{center}
}
    
\newenvironment{rnegocio}[1]
{
 \stepcounter{rncounter}

 \begin{center}
 \begin{tabular}{|l|p{0.8\linewidth}|}
 \hline
 \multicolumn{2}{|p{\textwidth}|}
 {
  \large \textbf{#1 (RN\ifnum\value{rncounter}<10 0\fi\arabic{rncounter})}
 } \\
 \hline
 Descrição & 
}
{
 \\ \hline
 \end{tabular} 
 \end{center}
}


% --------------------
% Inicio do documento |
% --------------------
\begin{document}
% ----------------------
%          Capa         |
% ----------------------
\begin{titlepage} %iniciando a "capa"
\begin{center} %centralizar o texto abaixo
{\large Universidade Federal do Rio Grande do Norte }\\[0.2cm] %0,2cm é a distância entre o texto dessa linha e o texto da próxima
{\large Instituto Metrópole Digital}\\[0.2cm] % o comando \\ "manda" o texto ir para próxima linha
{\large Bacharelado em Tecnologia da Informação}\\[0.2cm]
{\large Análise e Projeto Orientado a Objetos}\\[5.1cm]
{\bf \huge Modelagem de Projeto}\\[5.1cm] % o comando \bf deixa o texto entre chaves em negrito. O comando \huge deixa o texto enorme
\end{center} %término do comando centralizar
{\large Aluno: Pablo Emanuell Lopes Targino}\\[0.7cm] % o comando \large deixa o texto grande
{\large Professor: Leonardo Cunha de Miranda}\\[5.1cm]
\begin{center}
{\large Natal}\\[0.2cm]
{\large 2018}
\end{center}
\end{titlepage} 
% ----------------------
%          Corpo        |
% ----------------------
\section{Introdução}

A modelagem de projeto tem o objetivo de fazer a estrutura básica de construção de um Software. São diversas técnicas utilizadas para cada fase de projeto. Esse documento contém os artefatos de software produzidos na fase de análise de projeto de um sistema de software orientado a objetos.

O sistema proposto é um aplicativo de divulgação de eventos e estabelecimento, onde pessoas podem criar anúncios que podem ser vistos pelo usuário e compartilhados em grupos. O objetivo principal é que o usuário do aplicativo tenha oportunidade de utilizar um software que pode recomendar lugares para sair e se divertir na sua cidade e no seu bairro.

\section{Descrição do projeto com as telas do sistema} 

A ideia consiste em projetar um aplicativo de divulgação de eventos e novos estabelecimentos, por meio de cadastros feitos pelo usuário.

O aplicativo foi feito pensando em pessoas que não sabem lugares para sair (Ex.: novos moradores de um determinado local), tem um smartphone e tem acesso à internet. Assim essas pessoas tem fácil acesso a esse tipo de informação, evitando que seja gasto tempo procurando em ferramentas diversas que não são especificas para tal finalidade.

O ambiente de uso do aplicativo é um lugar qualquer, já que basta ter acesso à internet para usufruir da ferramenta.

Para desenhar as interfaces utilizei o Balsamiq, um software para construção de protótipos de baixa fidelidade.

Quando o usuário assumi o papel de divulgador de eventos/estabelecimentos, suas atribuições são:\medskip
\begin{easylist}[itemize]
& Divulgar um evento/estabelecimento.
& Gerenciar seu(s) eventos/estabelecimento.
& Ver histórico de eventos realizados por ele.
\end{easylist}\medskip
Ao divulgar um evento/estabelecimento o usuário deve dizer o que quer divulgar e fornecer o nome do evento, o horário, a localização, uma imagem do evento, uma indicação se o evento é público ou privado e, opcionalmente, observações sobre o evento. As telas que ele usurá para fazer isso serão parecidas com essa:

\telasmart{tela_cadastro_evento}{Tela de cadastro de evento ou estabelecimento.}

Ao final do procedimento o evento estará disponível para que os outros usuários que tenham acesso a ele vejam e possam sinalizar sua participação e compartilhar o evento.

Uma vez divulgado o evento ele poderá ser gerenciado - cancelado ou editado. A tela para gerenciar o estabelecimento será a mesma usada para gerenciar os eventos. Veja o protótipo na figura \ref{fig:gerenciar_evento}:

\telasmart{gerenciar_evento}{Telas de edição de evento e de  estabelecimento.}

A quantidade de eventos de cada tipo que cada usuário pode cadastrar se encontra nas regras de negócio  RN02 e RN03 na seção 4 desse documento.

O histórico de eventos realizados pelo divulgador estará disponível ao divulgar seu primeiro evento. Ao acessar esse menu, o usuário poderá ver todas as informações do eventos que já realizou. A tela onde as duas últimas operações serão ativadas será um menu drop-down que será apresentado junto com as operações do papel a seguir. A tela do histórico em si é essa:

\telasmart{eventos_criados}{Tela de histórico de eventos realizados por usuário.}

Os eventos divulgados no sistema serão usados por pessoas que estão procurando lugares para sair, nesse papel as atribuições do usuário serão as seguintes:\medskip
\begin{easylist}[itemize]
& Ver eventos disponíveis na região.
& Favoritar categorias de evento.
& Ver agenda de eventos.
& Ver histórico de eventos.
\end{easylist}\medskip

Antes de apresentar as próximas telas é importante ressaltar onde boa parte dos recursos do aplicativo serão engatilhados, esse lugar é um menu acordeão (dropdown) que está disponível na tela principal do App. A figura que mostra esse menu é:

\telasmart{menu_home}{Menu com várias operações do sistema.}

A opção minha conta já tem protótipo definido, mas ela não será mostrada ainda, por se tratar de uma funcionalidade que pode ser projetada posteriormente para se adaptar ao nível do sistema nas próximas fases.

Para ver os eventos, o usuário se utilizará de um mapa com coordenadas pré-configuradas para seu local (ou para o evento mais próximo). Os pontos no mapa, possivelmente, terão o mesmo tamanho, pois não queremos que um evento seja mais chamativo que outro. Os eventos que apareceram nesse tela são delimitados pela regra de negócio RN05:

\telasmart{mapa}{Mapa de eventos de interesse do usuário}			

As categorias favoritas vão servir para cada usuário ser notificado de eventos recentes que eles poderiam gostar. Na aba principal ele poderá ver todas as categorias e marcar as que ele mais gosta, como pode ser visto na imagem: 

\telasmart{fav}{Tela principal com as categorias}

Como o aplicativo é de eventos,  é natural que o usuário queira ter alguma maneira de se lembrar de onde tem que ir. Para isso iremos dispor de uma agenda que conterá as datas do eventos que ele marcou presença. Além disso o aplicativo irá disparar uma notificação em determinado tempo antes do evento. A tela será baseado no protótipo abaixo:

\telasmart{agenda}{Agenda de eventos do usuário}

Também é comum que uma pessoa queira se lembrar de onde ela foi determinado dia, essa funcionalidade estará disponível no sistema. A imagem abaixo demonstra como isso será feito:

\telasmart{hist_eventos}{Histórico de eventos que o usuário participou (Lembranças).}
 
\section{Casos de uso} \bigskip

Os casos de uso foram feitos com base no recursos essenciais do sistema, o diagrama foi feito utilizando a ferramenta Astah. O DCU está contido na imagem da próxima página.

\diagrama{usecase}{Diagrama de casos de uso.}
\pagebreak

\textcolor{red}{
	Contudo, com o surgimento de dificuldades ao modelar as VCP's do diagrama de classes, o diagrama de casos de uso sofreu alterações. Além disso, devido ao aumento na divisão dos casos de uso, o sistema foi dividido em três pacotes, como pode ser visto nas figura \ref{fig:usecase2}.
}
	\diagrama{usecase2}{Diagrama de casos de uso melhorado.}
	\diagrama{usecase21}{Diagrama do modulo principal.}
	\diagrama{usecase22}{Diagrama do modulo de divulgador.}
	\diagrama{usecase23}{Diagrama do modulo de grupos.}
	
\textcolor{red}{
	Como pode ser visto, também foi removido o ator sistema de notificação pois ele não precisa ser representado como ator para entender o funcionamento do sistema.

Alguns casos de uso estão documentados usando uma descrição numerada extendida com grau de detalhamento essencial, isso tudo está descrito abaixo:
} 
\textcolor{red}{
\begin{casosdeuso}{Divulgar evento}
 \sumario{Usuário cadastra um novo evento ao sistema.}
 \ator{Divulgador.}
 \precond{Usuário está autentificado.}
\fluxo
\begin{enumerate}[itemsep=0mm]
 \item Usuário solicita o cadastro de um novo Anúncio.
 \item Sistema Pergunta qual o tipo de anúncio.
 \item Usuário informa que quer criar um evento
 \item Sistema exibe formulário pedindo as informações que descrevem o evento.
 \item Usuário preenche as informações e solicita o envio do formulário.
 \item Sistema pergunta a categoria do evento.
 \item Usuário escolhe a categoria.
 \item Sistema cadastra novo evento, notifica usuários próximos e o caso de uso termina.
\end{enumerate}
\begin{fluxoe}{3}{Violação de RN02}
 \item Em caso de violação de RN02, o sistema irá alertar o usuário e interromper a operação. \end{fluxoe}
\begin{fluxoe}{3}{Violação de RN04}
 \item Em caso de violação de RN04, o sistema irá alertar o usuário e interromper a operação. \end{fluxoe}
\end{casosdeuso}
}

\textcolor{red}{
\begin{casosdeuso}{Adicionar membros}
 \sumario{O administrador do grupo coloca mais pessoas para participar dele.}
 \ator{Administrador de grupos.}
 \precond{Usuário está autentificado e é administrador do grupos}
\fluxo
\begin{enumerate}[itemsep=0mm]
 \item Usuário solicita os grupos que ele participa.
 \item Sistema exibe os grupos.
 \item Usuário seleciona um grupo em que ele é administrador.
 \item Sistema exibe tela do grupo.
 \item Usuário seleciona a opção de adicionar novos membros.
 \item Sistema exibe tela com campo para colocar o nome do usuário que será adicionado.
 \item Usuário cadastra todos os membros e solicita a confirmação ao sistema.
  \item Sistema cadastra os novos membros e o caso de uso termina. 
\end{enumerate}
\begin{fluxoe}{7}{Membro existente} 
 \item Se o membro já pertence ao grupo, o sistema notificará ao usuário. 
\end{fluxoe}
\end{casosdeuso}
}

\textcolor{red}{
\begin{casosdeuso}{Editar evento}
 \sumario{Usuário altera as informações de um evento.}
 \ator{Divulgador.}
 \precond{Usuário está autentificado e já criou no mínimo um evento.}
\fluxo
\begin{enumerate}[itemsep=0mm]
 \item Usuário solicita ao sistema eventos criados por ele.
 \item Sistema exibi uma lista de eventos anunciados usuário.
 \item Usuário seleciona um dos eventos.
 \item Sistema exibe as informações sobre o evento.
 \item Usuário solicita a alteração dessas informações.
 \item Sistema exibe formulário com as informações que podem ser alteradas.
 \item Usuário edita as informações do formulário e solicita a confirmação.
 \item Sistema cadastra as novas informações, notifica os usuário participante do evento e o caso de uso termina.
\end{enumerate}
\begin{fluxoa}{5}{Notificação de alteração de evento}
 \item Conforme RN01, os usuário que marcaram participação serão notificados da alteração.
\end{fluxoa}
\end{casosdeuso}
}

\textcolor{red}{
\begin{casosdeuso}{Participar de  eventos}
 \sumario{Usuário visualiza eventos próximos a ele.}
 \ator{Usuário procurando eventos.}
 \precond{Usuário está autentificado e existem eventos próximos a ele.}
\fluxo
\begin{enumerate}[itemsep=0mm]
 \item Usuário solicita eventos na proximidade.
 \item Sistema exibe os eventos próximos.
 \item Usuário escolhe um dos eventos.
 \item Sistema exibe informações sobre o evento.
 \item Usuário decide se vai participar do evento.
 \item Sistema cadastra participação no evento, se o usuário desejar, e o caso de uso termina.
 \end{enumerate}
 \begin{fluxoe}{4}{Usuário já participa de evento}
 \item Se o usuário já estiver participando, o botão de participação aparecerá diferenciado.
 \item Se o usuário clicar no botão de participação nessa condição, o sistema irá perguntar se ele deseja cancelar a participação no evento.
 \end{fluxoe} 
\end{casosdeuso}
}


\textcolor{red}{
\begin{casosdeuso}{Ver agenda}
 \sumario{Usuário visualiza agenda mensal com eventos marcados.}
 \ator{Usuário procurando eventos.}
 \precond{Usuário está autentificado.}
 \fluxo
\begin{enumerate}[itemsep=0mm]
 \item Usuário solicita sua agenda.
 \item Sistema exibe calendário com datas de participação em um evento.
 \item Usuário seleciona um data.
 \item Sistema exibi lista com eventos que o usuário participou ou participará naquela data.
 \item Usuário seleciona um dos eventos ou volta para o item 2.
 \item Sistema exibe informações do evento.
 \item Usuário voltar e o caso de uso voltar para o item 4 ou o caso de uso termina
\end{enumerate}
\end{casosdeuso}
}

\textcolor{red}{
\begin{casosdeuso}{Criar grupo}
 \sumario{Usuário cria um grupo para compartilhar eventos}
 \ator{Dono de grupo}
 \precond{Usuário está autentificado.}
\fluxo
\begin{enumerate}[itemsep=0mm]
 \item Usuário solicita criação do grupo.
 \item Sistema exibe formulário com informações exigidas na criação do grupos.
 \item Usuário fornece as informações.
 \item Sistema pergunta quais categorias serão permitidas.
 \item Usuário seleciona as categorias e confirma.
 \item Sistema exibe tela para adicionar membros com campo para colocar o nome do usuário e botão para adicionar mais um.
 \item Usuário cadastra todos os membros e solicita a confirmação ao sistema.
 \item Sistema cadastra o grupo e notifica os todos os membros e o caso de uso termina.
\end{enumerate}
\end{casosdeuso}
}

\textcolor{red}{
\begin{casosdeuso}{Excluir grupo}
 \sumario{Usuário deleta grupo.}
 \ator{Dono de grupo.}
 \precond{Usuário está autentificado e é dono do grupo.}
\fluxo
\begin{enumerate}[itemsep=0mm]
 \item Usuário solicita lista de grupos em que ele participa.
 \item Sistema exibe os grupos.
 \item Usuário seleciona grupo em que ele é dono.
 \item Sistema tela principal do grupo.
 \item Usuário Solicita a remoção do grupo.
 \item Sistema pergunta se usuário que mesmo realizar a operação.
 \item Usuário confirma a solicitação.
 \item Sistema remove o grupo e todos as postagens, notifica todos os membros e o caso de uso termina.
\end{enumerate}
\end{casosdeuso}
} 
 
\textcolor{red}{
\begin{casosdeuso}{Excluir categoria}
 \sumario{Administrador do sistema excluí uma categoria.}
 \ator{Administrador do sistema.}
 \precond{Usuário está autentificado e é administrador do sistema.}
\fluxo
\begin{enumerate}[itemsep=0mm]
 \item Administrador solicita lista de categoria.
 \item Sistema exibe todas as categorias.
 \item Admin seleciona uma categoria  para exclusão.
 \item Sistema pergunta a categoria substituta.
 \item Usuário informa a categoria.
 \end{enumerate}
 \begin{fluxoa}{5}{Politica de RN08}
  \item Para cada Anuncio pertencente à categoria antiga, o sistema substituirá para a categoria nova.
  \end{fluxoa}
\end{casosdeuso}
}

\textcolor{red}{
\begin{casosdeuso}{Postar}
 \sumario{Membro cria postagem mostrando um evento.}
 \ator{Membro do grupo.}
 \precond{Usuário está autentificado e é membro do grupo.}
\fluxo
\begin{enumerate}[itemsep=0mm]
 \item Usuário solicita eventos disponíveis.
 \item Sistema exibe os eventos.
 \item Usuário escolhe um dos eventos.
 \item Sistema exibe informações sobre o evento.
 \item Usuário solicita compartilhamento do evento.
 \item Sistema exibe lista de grupos em que ele participa.
 \item Usuário seleciona um ou mais grupos que ele deseja compartilhar o evento e confirma a operação.
 \item Sistema cria postagem nos grupos selecionados, notifica seus respectivos membros e o caso de uso termina.
\end{enumerate}
\end{casosdeuso}
}

\textcolor{red}{
\begin{casosdeuso}{Excluir Anúncio}
 \sumario{Administrador do aplicativo exclui anúncio.}
 \ator{Administrador do aplicativo.}
 \precond{Usuário está autentificado e é administrador do aplicativo.}
\fluxo
\begin{enumerate}[itemsep=0mm]
 \item Admin solicita listagem de anúncios.
 \item Sistema retorna a lista.
 \item Admin seleciona o anúncio.
 \item Sistema apresenta operações possíveis.
 \item Admin escolhe excluir o anúncio.
 \item Sistema remove o anúncio do sistema e o caso de uso termina.
\end{enumerate}
\begin{fluxoe}{6}{Politica de RN09}
\item O anunciante deverá ser alertado e em caso de o anúncio ser um evento, de acordo com RN09, Todos os participantes serão notificados.
\end{fluxoe}
\end{casosdeuso}
}

\section{Regras de negócio} 

\par As regras de negócio também foram feitas, mas a maior parte delas ainda não foi pensada. As regras podem ser conferidas abaixo:

\begin{rnegocio}{Cancelamento ou alteração de evento}
Ao editar ou cancelar um evento todos os usuários participantes serão notificados.
\end{rnegocio}

\begin{rnegocio}{Restrição do número de eventos}
Cada usuário só pode cadastrar 2 eventos ao mesmo tempo.
\end{rnegocio}

\begin{rnegocio}{Restrição do número de estabelecimentos}
Cada usuário só pode divulgar um estabelecimento.
\end{rnegocio}

\begin{rnegocio}{Restrição de local do evento}
Ao criar um evento em um local que já foi criado, o sistema irá gerar um alerta notificando que já existe um evento naquele ambiente.
\end{rnegocio}

\begin{rnegocio}{Restrição de eventos no mapa}
Os eventos que aparecem no mapa pertecem às categorias favoritados e obedecem a um raio de alcance determinado pelo usuário.
\end{rnegocio}

\begin{rnegocio}{Restrição de horário de funcionamento}
Um estabelecimento pode ter no máximo 4 horários de funcionamento.
\end{rnegocio}

\begin{rnegocio}{Restrição sobre conteúdo publicado no grupo}
O administrador do grupo pode restringir a categoria dos eventos publicados no grupo.
\end{rnegocio}

\begin{rnegocio}{Politica de exclusão de categoria}
Ao excluir uma categoria, o administrador terá que escolher uma outra na qual todos os eventos pertencentes à categoria antiga deverão ser vinculados.
\end{rnegocio}

\begin{rnegocio}{Politica de exclusão de Anúncio}
Ao excluir um Anúncio de qualquer tipo, o dono deverá ser notificado. Se o tipo do anúncio for um evento todos os participantes serão notificados, e as postagens feitas com o anúncio indicarão que o o evento foi excluído.
\end{rnegocio}

	\section{Classes de domínio} \bigskip
\textcolor{red}{As classes de domínio do sistema foram feitas com base nos casos de uso e para suportar as regras de negócio. 
Foram removidas diversas coisas e alteradas durante o novo ciclo de analise. O diagrama pode ser visto na figura \ref{fig:Diagrama}}.
 
\diagrama{Diagrama}{Diagrama de classes do sistema.} 

Além desse digrama de classes, foram feitos 8 VCP'S para os 8 dos casos de uso apresentados. A maior dificuldade foi achar casos de uso que interajam com muitas classes, isso aconteceu porque o sistema foi pensando em 2 partes diferentes: os grupos, os anúncios.
Todos os vcp's se encontram nas imagens a seguir: 


\diagrama{vcp_adicionarmembros}{VCP para o caso de uso adicionar membros.}
\diagrama{vcp_criargrupo}{VCP para o caso de uso criar grupo}
\diagrama{vcp_editarevento}{VCP para o caso de uso editar evento.}
\diagrama{vcp_excluircategoria}{VCP para o caso de uso excluir categoria.}
\diagrama{vcp_excluirgrupo}{VCP para o caso de uso excluir grupo.}
\diagrama{vcp_participarevento}{VCP para o caso de uso participar evento.}
\diagrama{vcp_postar}{VCP para o caso de uso postar.}
\diagrama{vpc_divulgarevento}{VCP para o caso de uso divulgar evento.}

\pagebreak


As classes não foram formadas a partir da identificação dirigida ás responsabilidades, no entanto, isso ajudou na identificação ou concretização das classes. Os cartões CRC podem ser conferidos abaixo:

\textcolor{red}{
\begin{center}
 \begin{tabular}{|p{0.5\linewidth}|p{0.5\linewidth}|}
  \hline
  \multicolumn{2}{|p{\textwidth}|}{
   {\large \textbf{Usuário}}} \\
  \hline
  \textbf{Responsabilidades} & \textbf{Colaboradores} \\ 
  \hline
  Saber seu nome &  Agenda \\
  \hline
  Manter sua agenda & Evento \\
  \hline
  Manter categorias favoritas & Categoria \\
  \hline
  Saber os Eventos divulgados & Grupo  \\
  \hline
  Saber os estabelecimentos divulgados &  \\
  \hline
  Saber suas denúncias feitas &  \\
  \hline
  Saber eventos que participou/participará &  \\
  \hline
  Saber grupos em que participa & \\  
  \hline
 \end{tabular} 
\end{center}
}
    % ----------------------------
\textcolor{red}{
    \begin{center}
   	 \begin{tabular}{|p{0.5\linewidth}|p{0.5\linewidth}|}
\hline
 	\multicolumn{2}{|p{\textwidth}|}{
{\large \textbf{Anúncio}}
}  \\
\hline
\textbf{Responsabilidades} & \textbf{Colaboradores} \\ 
\hline
  	Saber seu local & Categoria  \\
  	\hline
  	Saber sua categoria & Usuário \\
  	\hline
  	Saber seu anunciante &  \\
  	\hline
   	\end{tabular} 
    \end{center}
}
    
     % ----------------------------
\textcolor{red}{
    \begin{center}
   	 \begin{tabular}{|p{0.5\linewidth}|p{0.5\linewidth}|}
\hline
 	\multicolumn{2}{|p{\textwidth}|}{
{\large \textbf{Evento}}
}  \\
\hline
\textbf{Responsabilidades} & \textbf{Colaboradores} \\ 
\hline
  	Saber sua data & Usuário \\
  	\hline
  	Saber seu horário &  \\
  	\hline
  	Saber seus participantes & \\
  	\hline
   	\end{tabular} 
    \end{center}
}
    % ----------------------------
\textcolor{red}{
    \begin{center}
   	 \begin{tabular}{|p{0.5\linewidth}|p{0.5\linewidth}|}
\hline
 	\multicolumn{2}{|p{\textwidth}|}{
{\large \textbf{Estabelecimento}}
}  \\
\hline
\textbf{Responsabilidades} & \textbf{Colaboradores} \\ 
\hline
  	Saber seu horário de funcionamento & Usuário \\
  	\hline
  	Saber seu dono(a) &  \\
  	\hline
   	\end{tabular} 
    \end{center}
}
    % ----------------------------
\textcolor{red}{
    \begin{center}
   	 \begin{tabular}{|p{0.5\linewidth}|p{0.5\linewidth}|}
\hline
 	\multicolumn{2}{|p{\textwidth}|}{
{\large \textbf{Categoria}}
}  \\
\hline
\textbf{Responsabilidades} & \textbf{Colaboradores} \\ 
\hline
  	Saber seu título &  \\
  	\hline
   	\end{tabular} 
    \end{center}
}
    % ----------------------------
\textcolor{red}{
    \begin{center}
   	 \begin{tabular}{|p{0.5\linewidth}|p{0.5\linewidth}|}
\hline
 	\multicolumn{2}{|p{\textwidth}|}{
{\large \textbf{Agenda}}
}  \\
\hline
\textbf{Responsabilidades} & \textbf{Colaboradores} \\ 
\hline
  	Saber seus eventos & Evento \\
  	\hline
   	\end{tabular} 
    \end{center}
}
     % ----------------------------
\textcolor{red}{
    \begin{center}
   	 \begin{tabular}{|p{0.5\linewidth}|p{0.5\linewidth}|}
\hline
 	\multicolumn{2}{|p{\textwidth}|}{
{\large \textbf{Grupo}}
}  \\
\hline
\textbf{Responsabilidades} & \textbf{Colaboradores} \\ 
\hline
  	Manter seus membros & Usuário \\
  	\hline
  	Manter seus administradores & Postagem \\
  	\hline
  	Manter suas postagens & \\
  	\hline
   	\end{tabular} 
    \end{center}
}
    % ----------------------------
\textcolor{red}{
    \begin{center}
   	 \begin{tabular}{|p{0.5\linewidth}|p{0.5\linewidth}|}
\hline
 	\multicolumn{2}{|p{\textwidth}|}{
{\large \textbf{Denúncia}}
}  \\
\hline
\textbf{Responsabilidades} & \textbf{Colaboradores} \\ 
\hline
  	Saber seu motivo &  Data \\
  	\hline
  	Saber sua data &  Horário \\
  	\hline
  	Saber seu horário &  Anúncio \\
  	\hline
  	Saber seu anuncio denunciado & Usuário \\
  	\hline
  	Saber seu denunciante &  \\
  	\hline
   	\end{tabular} 
    \end{center}
}
    % ----------------------------
\textcolor{red}{
\begin{center}
 \begin{tabular}{|p{0.5\linewidth}|p{0.5\linewidth}|}
  \hline
  \multicolumn{2}{|p{\textwidth}|}{
   {\large \textbf{Postagem}}} \\
  \hline
  \textbf{Responsabilidades} & \textbf{Colaboradores} \\ 
  \hline
  Manter seu texto &  Usuário \\
  \hline
  Saber seu usuário & Anúncio \\
  \hline
  Saber seu anúncio & Data \\
  \hline
  Saber sua data & Horário \\
  \hline
  Saber seu horário &  \\
  \hline 
 \end{tabular} 
\end{center}
}

\section{Diagramas de interação}

Para os 8 casos de uso escolhidos para as VCPS também foram feitos os diagramas de sequencia e comunicação. Seguindo na sequencia: 2 diagramas de comunicação exclusivos, 2 diagramas de sequencia exclusivos, 4 pares de diagrama de sequencia e comunicação:

\diagrama{com_criargrupo1}{Diagrama de comunicação do caso de uso criar grupos}
\diagrama{com_excluirCategoria1}{Diagrama de comunicação do caso de uso excluir categoria}
\diagrama{seq_adicionarmembros1}{Diagrama de sequencia do caso de uso adicionar membros}
\diagrama{seq_excluirgrupo1}{Diagrama de sequencia do caso de uso excluir grupo}

\diagrama{com_divulgarevento2}{Diagrama de comunicação do caso de uso divulgar evento}
\diagrama{seq_divulgarevento2}{Diagrama de sequencia do caso de uso divulgar evento}

\diagrama{com_editarevento2}{Diagrama de comunicação do caso de uso editar evento}
\diagrama{seq_editarevento2}{Diagrama de sequencia do caso de uso evento}

\diagrama{com_participarevento2}{Diagrama de comunicação do caso de uso participar de evento}
\diagrama{seq_participarevento2}{Diagrama de sequencia do caso de uso participar de evento}

\diagrama{com_postar2}{Diagrama de comunicação do caso de uso postar}
\diagrama{seq_postar2}{Diagrama de sequencia do caso de uso postar}

\section{Glossário de conceitos}
    
\begin{enumerate}[label=\textbf{\arabic*}]
\item \textbf{Usuário:} usuário do aplicativo.
\item \textbf{Anúncio:} divulgação de evento ou estabelecimento.
\item \textbf{Evento:} tipo de divulgação permitida.
\item \textbf{Local:} Endereço de um lugar.
\item \textbf{Data:} dia, mês e ano de um acontecimento.
\item \textbf{Horário:} horas e minutos de um acontecimento.
\item \textbf{Estabelecimento:} comercio, instituição ou outro tipo de entidade que pode reunir pessoas.
\item \textbf{Categoria:} Forma de rotular os anúncios para o usuário.
\item \textbf{Agenda:} Tela que reúne o conjunto de eventos que o usuário quer participar em cada data.
\item \textbf{Grupo:} Ferramenta que reúne usuários para compartilhar anúncios de determinadas categorias.
\item \textbf{Membro:} usuário que pertence a algum grupo.
\item \textbf{Postagem:} Anúncio com texto compartilhado em um grupo.
\item \textbf{Denúncia:} Relato de abuso feito por um usuário para um anúncio.
\end{enumerate}
\end{document}